\documentclass[a4paper,12pt]{report}
\usepackage{alltt, fancyvrb, url}
\usepackage{graphicx}
\usepackage[utf8]{inputenc}
\usepackage{float}
\usepackage{hyperref}
\usepackage[italian]{babel}
\usepackage[italian]{cleveref}
\title{Progetto di IOT
    \\ SmartBin}

\author{Scorza Edoardo 0001077424 \\ Giorgini Matteo 0001136576 \\ Giuseppe Argentiere }
\date{1 dicembre 2024}   
\begin{document}
\maketitle
\tableofcontents
\chapter{Hardware}
La implementazione fisca del circuito è realizzata con i componenti 
richesti dalla specifica:
\begin{itemize}
    \item \textbf{LCD I2C}
    \begin{itemize}
        \item SDA: A4
        \item SCL: A5
    \end{itemize}
    \item \textbf{Bottone Open}
    \begin{itemize}
        \item Pin: 6
    \end{itemize}
    \item \textbf{Bottone Close}
    \begin{itemize}
        \item Pin: 7
    \end{itemize}
    \item \textbf{Passive Infrared (PIR Sensor)}
    \begin{itemize}
        \item Pin: 2
    \end{itemize}
    \item \textbf{Sonar}
    \begin{itemize}
        \item Trig: 4
        \item Echo: 5
    \end{itemize}
    \item \textbf{Red LED}
    \begin{itemize}
        \item Pin: 8
    \end{itemize}
    \item \textbf{Green LED}
    \begin{itemize}
        \item Pin: 9
    \end{itemize}
    \item \textbf{Temp Sensor (LM35)}
    \begin{itemize}
        \item Pin: A3
    \end{itemize}
    \item \textbf{Servo Motor}
    \begin{itemize}
        \item Pin: 3
    \end{itemize}
\end{itemize}
\chapter{Software}
Per la realizzazione del software abbiamo optato per un sistema di task e FSM,
inizialmente prevedavamo l'uso di una libreria, ma, il mancato supporto di Functional
ci avrebbe impedito di realizzare un oggetto con dentro la FSM.
\section{Sheduler}
Per lo scheduler abbiamo usato la base trovata nel codice del corso e lo abbiamo 
modificato per supportare la comunicazione di task
\section{Task}
La classe Task è una versione modificata di quella base, con un riferimento alla propria 
variabile condivisa e un accesso a quelle delle altre task
\section{FSM}
Per semplicità e unicità delle specifiche, la macchina a stati finiti
è realizzata con uno switch, nella quale sono definite transizioni, chiamate in entrata, uscita
e Timeout
\section{Struttura}
Il progetto è separato in varie task, suddivise in due categorie:
\begin{itemize}
    \item \textbf{Task di Report} \\
    Queste task si occupano di leggere e/o passare dati:
    \begin{itemize}
        \item \textbf{ButtonTask}
        \item \textbf{TemperatureTask} (FSM)
        \item \textbf{GuiTask}
        \item \textbf{WasteDetectorTask}
        \item \textbf{UserDetectorTask}
    \end{itemize}
    
    \item \textbf{Task Decisionali} \\
    Queste task operano sull'hardware:
    \begin{itemize}
        \item \textbf{BinTask} (FSM)
    \end{itemize}
\end{itemize}
\end{document}